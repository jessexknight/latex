\documentclass[12pt]{article}
% packages
\usepackage[margin=3cm]{geometry} % page margins, etc.
\usepackage[colorlinks]{hyperref} % clickable + coloured links
\usepackage{mathtools} % various tools for math
\usepackage{graphicx} % embedding graphics
\usepackage{siunitx} % dealing with scientific units + 'S' column type
\usepackage{multicol,multirow} % joining table cells
\usepackage{booktabs} % better layout of tables
\usepackage{authblk} % for multiple authors + affils
\usepackage[style=numeric]{biblatex} % citation references
\graphicspath{{./}{./fig/}} % where to look for figures
\addbibresource{main.bib} % where to look for citation info
% spacing tweaks
\setlength{\parskip}{1ex} % space between paragraps
\setlength{\parindent}{0pt} % indent length
\setlength{\abovecaptionskip}{1ex} % space above captions
\setlength{\belowcaptionskip}{1ex} % space below captions
% titleblock info
\title{A very ineresting result}
\author[1,2]{Jesse Knight}
\affil[1]{Main Institution}
\affil[2]{Another Institution}
\date{\today}
% output begins
\begin{document}
  \maketitle % print the title block
  %============================================================================
  \section{Introduction}\label{intro}
  \subsection{Background}\label{intro.back}
  In the beginning \dots % first par in *section not indented
  \par
  A new paragraph begins. % all other pars are indented
  We reference Section~\ref{intro.back}. % '\ref'erence a \label
  \par
  And another \cite{Last2023}. % cite a bibliographic entry using it's ID (usually AuthorYear)
  %============================================================================
  \section{Methods}\label{meth}
  Let $x = \sum_i x_i$. % inline math (different font, usually italics)
  In \eqref{eq:yi}, we see that \dots % ref Equation (1) overall
  specifically, \eqref{eq:yi.pi} shows \dots\ % ref Equation (1a)
  \begin{subequations} % automatically label sub-equations as (1a) etc.
    \label{eq:yi} % label (1) overall
    \begin{align} % align sub-equations at '&'
      y_i &= \sum_{ij} \beta_j x^2_j \\ % sub/super-scripts
          &= \left[ \begin{matrix} a & b \\ c & d \end{matrix} \right] \\ % matrices + auto-size "[...]"
          &= 2 \pi \label{eq:yi.pi}% greek symbols
    \end{align}
  \end{subequations}
  %============================================================================
  \section{Results}\label{res}
  % figures often contain graphics, tables often contain tabular content
  % but these are all different concepts and don't have to go together
  In Figure~\ref{fig:latex}, we see \dots % another cross-reference
  \begin{figure} % figures 'float' to avoid breaking up text
    % control the placement "gently" using [h] = here, [t] = page top, [b] = page bottom
    \centering % center everything within current scope (figure environment)
    \includegraphics[width=.5\linewidth]{latex}% <- this comment suppresses extra space
    \includegraphics[width=.5\linewidth]{latex}
    \caption{THE FIGURE CAPTION} % caption the figure
    \label{fig:latex} % label the figure
  \end{figure}
  \par
  Table~\ref{tab:data} also shows \dots
  \begin{table}[h] % table is another kind of 'float'
    \centering
    \caption{THE TABLE CAPTION}
    \label{tab:data}
    \begin{tabular}{lSS} % tabular environment needs argument: types of columns
      % 'S' column type provided by siunitx aligns on decimals
      \toprule % add a thick top line
      % '&' moves to the next cell and \\ moves to the next row
      Parameter & \multicolumn{2}{c}{Experiment} \\ % multicolumn joins cells horizontally
      \cmidrule(rl){2-3} % a thinner middle line just spanning cols 2-3
      % (rl) shortens the line a bit to avoid joining adjacent cmidrules
                &   A   & B     \\ % 3 cells: blank, A, B
      \midrule % a thinner middle line
      $\alpha$  & 10.2  & 12.8  \\
      $\beta$   &  0.17 &  0.43 \\
      \bottomrule % add a thick bottom line
    \end{tabular}
  \end{table}
  %============================================================================
  \printbibliography % as the name says ...
\end{document}
