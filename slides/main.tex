\PassOptionsToPackage{dvipsnames}{xcolor}
\documentclass[11pt,aspectratio=169]{beamer}
% metadata
\title[\LaTeX: A Short Introduction]{\LaTeX\\\textsc{\small{A Short Introduction}}}
\author{Jesse Knight}
\institute{MAP Centre for Urban Health Solutions}
\date{2019 April 24}
% ==============================================================================
% load packages with [options]
% hyperlinks
\usepackage[
  colorlinks,
  pdfusetitle]
  {hyperref}
% reference management
\usepackage[
  backend      = bibtex,
  style        = numeric,
  maxcitenames = 2,
  sorting      = none,
  url          = true,
  isbn         = false,
  doi          = true,
  eprint       = false]
  {biblatex}
% margins
\usepackage[margin=2cm]{geometry}
% syntax-highlight LaTeX code
\usepackage{listings}
% load packages without [options]
\usepackage{tikz,amsmath,nameref,float,booktabs}
% ==============================================================================
% package configuration
% choose hyperlink colors
\hypersetup{
  citecolor = purple,
  linkcolor = blue,
  urlcolor  = magenta}
% format LaTeX code
\lstset{
  language = [LaTeX]TeX,
  breaklines = true,
  basicstyle     = \color[RGB]{128,128,128}\tt\scriptsize,
  backgroundcolor= \color[RGB]{240,240,240},
  keywordstyle   = \color[RGB]{ 64,128,255},
  texcsstyle     =*\color[RGB]{ 64,128,255},
  identifierstyle= \color[RGB]{ 64, 64, 64},
  numberstyle    = \color[RGB]{ 64, 64, 64},
  morekeywords = {cmd,includegraphics,citeauthor,citeyear,section,subsection,subsubsection,printbibliography}
}
% where to look for figures
\graphicspath{{figs/},{figs/m/},{figs/tikz/}}
% tikz can produce plots & charts in LaTeX - load some style packages here
\usetikzlibrary{shapes,arrows.meta,calc,positioning}
% where to look for references
\bibliography{library.bib}
% redefining some spacing stuff
\setlength{\parindent}{0pt}
\setlength{\parskip}{8pt}
\setlength{\skip\footins}{32pt}
%%%%%%%%%%%%%%%%%%%%%%%%%%%%%%%%%%%%%%%%%%%%%%%%%%%%%%%%%%%%%%%%%%%%%%%%%%%%%%%%%%%%%%%%%%%%%%%%%%%%
\begin{document}
%%%%%%%%%%%%%%%%%%%%%%%%%%%%%%%%%%%%%%%%%%%%%%%%%%%%%%%%%%%%%%%%%%%%%%%%%%%%%%%%%%%%%%%%%%%%%%%%%%%%
\begin{frame}
  \maketitle
\end{frame}
%---------------------------------------------------------------------------------------------------
\begin{frame}[label=overview]{Overview}
  \tableofcontents
\end{frame}
%%%%%%%%%%%%%%%%%%%%%%%%%%%%%%%%%%%%%%%%%%%%%%%%%%%%%%%%%%%%%%%%%%%%%%%%%%%%%%%%%%%%%%%%%%%%%%%%%%%%
\section{Introduction}
%---------------------------------------------------------------------------------------------------
\begin{frame}{What is \LaTeX?}
  \pause
  A typesetting program:
  \pause
  \textit{content} $\rightarrow$ \textit{a document}
\end{frame}
%---------------------------------------------------------------------------------------------------
\begin{frame}{}
  \centering
  \begin{tabular}{m{0.4\linewidth} m{1cm} m{0.4\linewidth}}
    \parbox{\linewidth}{\centering input: \texttt{filename.tex}}
    & \LaTeX &
    \parbox{\linewidth}{\centering output: \texttt{filename.pdf}}
    \\[0.5em]
    \frame{\includegraphics[width=\linewidth]{io-in.png}}
    & \bigarrow{1,0} &
    \frame{\includegraphics[width=\linewidth]{io-out.pdf}}
  \end{tabular}
\end{frame}
%---------------------------------------------------------------------------------------------------
\begin{frame}{Advantages of \LaTeX}
  \pause
  \begin{itemize}
    \item<+-> separate content and formatting
    \item<+-> automate numbers and cross-references
    \item<+-> beautiful math
    \item<+-> comments and version control
    \item<+-> split-up large documents into parts
    \item<+-> it's free!
  \end{itemize}
\end{frame}
%---------------------------------------------------------------------------------------------------
\begin{frame}[fragile]{How does \LaTeX\ Work?}
  Three layers:
  \begin{enumerate}
    \pause\item\parbox{2.3cm}{``primitives''} -- e.g.\ \lstinline|\def\pi{3.14}| defines a macro \lstinline|\pi| to contain ``3.14''\\[0.2em]
         \pause\parbox{2.3cm}{+ ``kernel''}   -- e.g.\ tools to combine primitives
    \pause\item\parbox{2.3cm}{``classes''}    -- e.g.\ an article, which should have: title, author, sections, etc.\\[0.2em]
         \pause\parbox{2.3cm}{+ ``packages''} -- e.g.\ modify or extend a basic class
    \pause\item\parbox{2.3cm}{``document''}   -- e.g.\ this specific article (including content!)
  \end{enumerate}
\end{frame}
%%%%%%%%%%%%%%%%%%%%%%%%%%%%%%%%%%%%%%%%%%%%%%%%%%%%%%%%%%%%%%%%%%%%%%%%%%%%%%%%%%%%%%%%%%%%%%%%%%%%
\section{Getting Started}
%---------------------------------------------------------------------------------------------------
\begin{frame}[fragile]{Your First Document}
  \begin{minipage}{0.44\linewidth}
    \begin{lstlisting}
\documentclass{article}
% document header
\begin{document}
  % document content
  Hello World
\end{document}
    \end{lstlisting}
  \end{minipage}
  \pause
  \begin{minipage}{0.55\linewidth}
    But first: \hrefc{https://www.overleaf.com}{Overleaf}
    \pause
    $\rightarrow$ ``New Project''
  \end{minipage}
\end{frame}
%---------------------------------------------------------------------------------------------------
\begin{frame}{Document Elements}
  \pause
  \begin{itemize}
    \item<+-> title, author, date
    \item<+-> sections
    \item<+-> math
    \item<+-> cross-references \& table of contents
    \item<+-> citations \& bibliography
  \end{itemize}
\end{frame}
%---------------------------------------------------------------------------------------------------
\begin{frame}{Putting it all Together}
  \pause
  \begin{itemize}
    \item<+-> e.g.\ \hrefc{../examples/thesis/main.pdf}{a thesis}
    \item<+-> e.g.\ \hrefc{../examples/cv/main.pdf}{a CV}
    \item<+-> e.g.\ \hrefc{../examples/article/main.pdf}{an article}
  \end{itemize}
\end{frame}
%---------------------------------------------------------------------------------------------------
\begin{frame}[fragile]{Helpful Resources}
\newcommand{\linkbox}[2]{\parbox{3.5cm}{\hrefc{#1}{#2}}}
  \begin{itemize}
    \item \linkbox{https://www.overleaf.com}{Overleaf}
    -- Online \LaTeX\ writing application\\
    \item \linkbox{https://www.latex-tutorial.com/installation}{\LaTeX\ Install Guide}
    -- To install \LaTeX\ on your computer (offline)\\
    \item \linkbox{https://www.texstudio.org}{TeXstudio}
    -- Great editor for composing \LaTeX\ ``code'' (offline)\\
    \item \linkbox{https://tex.stackexchange.com/questions/9363}{\TeX\ Stack Exchange}
    -- Q \& A style how-to and debugging help
    \item \linkbox{https://wch.github.io/latexsheet/latexsheet.pdf}{\LaTeX\ Cheat Sheet}
    -- A really nice reference for common commands
  \end{itemize}
\end{frame}
%%%%%%%%%%%%%%%%%%%%%%%%%%%%%%%%%%%%%%%%%%%%%%%%%%%%%%%%%%%%%%%%%%%%%%%%%%%%%%%%%%%%%%%%%%%%%%%%%%%%
\end{document}
%%%%%%%%%%%%%%%%%%%%%%%%%%%%%%%%%%%%%%%%%%%%%%%%%%%%%%%%%%%%%%%%%%%%%%%%%%%%%%%%%%%%%%%%%%%%%%%%%%%%